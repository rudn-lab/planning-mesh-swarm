\documentclass{article}

\usepackage{amsmath}
\usepackage{amssymb}
\usepackage{paracol}
\usepackage[russian]{babel}
\usepackage[14pt]{extsizes} % для того чтобы задать нестандартный 14-ый размер шрифта
\usepackage[left=20mm, top=15mm, right=15mm, bottom=30mm, footskip=15mm]{geometry} % настройки полей документа
\usepackage[utf8]{inputenc}
\usepackage[style=gost-numeric]{biblatex}
\usepackage{graphicx}
\usepackage[hidelinks]{hyperref}
\usepackage{minted}
\usepackage{caption}

\setminted{fontsize=\footnotesize}
\captionsetup[listing]{name=Листинг}

\usepackage{changepage}

%https://tex.stackexchange.com/questions/403463/signature-on-latex-lines
\newcommand\signature{%
   \begin{minipage}[t]{5cm}
   \vspace*{1.5ex}  % leave some space above the horizontal line
   \hrule
   \vspace{1mm} % just a bit more whitespace below the line
   \centering
   \begin{tabular}[t]{c}
   \small{(подпись)}
   \end{tabular}
   \end{minipage}}

\usepackage{xcolor}

\newcommand{\emptydate}{<<\underline{\phantom{99}}>> \underline{\phantom{февралиюня}} \the\year{} г.}

\addbibresource{sources.bib}

\title{Разработка PDDL планировщика}
\author{Матюхин Григорий Васильевич}

\begin{document}

\begin{titlepage}

  \begin{center}
  \hfill \break
  \large{РОССИЙСКИЙ УНИВЕРСИТЕТ ДРУЖБЫ НАРОДОВ}\\
  \large{ИМЕНИ ПАТРИСА ЛУМУМБЫ}\\
  \normalsize{Факультет \underline{физико-математических и естественных наук}}\\ 
  \normalsize{Кафедра \underline{математического моделирования и искусственного интеллекта}}\\

  \vspace*{\fill}

  \begin{flushright}
    \large{Утверждаю}\\
    \normalsize{Заведующий кафедрой} 
    \normalsize{математического \\ моделирования и \\ искусственного интеллекта \\} 
    \underline{\phantom{signature signature}} Малых М.Д. \\
    <<\underline{\phantom{day}}>> \underline{\phantom{month month}} 2025 г.
    \end{flushright}
 
  \vspace*{\fill}
  \Large{\textbf{НАУЧНО-ИССЛЕДОВАТЕЛЬСКАЯ РАБОТА\\ на тему}}
  \\
  \underline{Разработка планировщика}
  \vspace*{\fill}
  
  \end{center}
   
   \begin{flushright}
    Выполнил \\
    Студент группы \underline{НПИбд-01-21}\\
    Студенческий билет \textnumero{}: \underline{1032211403} \\
    \underline{Матюхин Григорий Васильевич \phantom{signature}}\\ \vspace{0.5cm}
   \end{flushright}

   \vspace*{\fill}

   \begin{flushright}
    Руководитель \\
    \underline{Виноградов Андрей Николаевич \phantom{signature}}\\ \vspace{0.5cm}
   \end{flushright}
   
  \begin{center} \textbf{МОСКВА} \\ 2025 г. \end{center}
  \thispagestyle{empty} % выключаем отображение номера для этой страницы
   
\end{titlepage}
    
\newpage

\tableofcontents

\newpage

\section{Введение}

Настоящяя работа посвящёна описанию реализации планировщика, предназначенного для построения планов в рамках формального языка PDDL. Рассматриваются ключевые архитектурные решения, структура кода и используемые алгоритмические подходы. Основное внимание уделяется внутреннему представлению состояний и действий, механизму подстановки переменных, а также обработке логических выражений и кванторов. Целью данной реализации является создание гибкой и расширяемой платформы для последующих экспериментов и модификаций.

\section{Предварительная информация}

\subsection{Планировщики}
Планировщики~\cite{ghallab2004automated} --- это системы, используемые для автоматического создания последовательности действий, необходимых для достижения заданной цели в определенной среде. Они широко применяются в различных областях, таких как робототехника, логистика, искусственный интеллект и управление задачами. Основная задача планировщика заключается в нахождении оптимального или приемлемого плана, который переводит систему из начального состояния в целевое состояние, удовлетворяя при этом заданным ограничениям и условиям.

\subsection{PDDL (Planning Domain Definition Language)}

PDDL (Planning Domain Definition Language)~\cite{mcdermott1998pddl}~\cite{gerevini2006pddl3}
--- это язык описания задач планирования, используемый в области искусственного интеллекта.
Он позволяет формализовать модель мира и определить цели, которых необходимо достичь.
Обычно описание задачи в PDDL состоит из двух основных файлов:
\textit{файла домена} и \textit{файла задачи}.
Это деление позволяет отделить общую модель мира (правила, действия, объекты и свойства)
от конкретной задачи, которая решается в рамках этого мира.

\begin{enumerate}
\item \textbf{Файл домена} описывает, какие действия возможны в рассматриваемом мире,
  какие предикаты используются для описания состояния,
    и как действия изменяют это состояние.
    Ниже приведён пример PDDL-домена для маленького склада:

  \begin{minted}{lisp}
(define (domain mini-warehouse)
  (:requirements :strips :typing :quantified-preconditions)
  (:types robot box location)
  (:predicates
    (at ?r - robot ?l - location)
    (box-at ?b - box ?l - location)
    (carrying ?r - robot ?b - box)
    (clean ?l - location)
    (adjacent ?l1 - location ?l2 - location) )
  (:action move
    :parameters (?r - robot ?from - location ?to - location)
    :precondition (and (at ?r ?from) (adjacent ?from ?to))
    :effect (and (not (at ?r ?from)) (at ?r ?to)) )
  (:action pick-up
    :parameters (?r - robot ?b - box ?l - location)
    :precondition (and (at ?r ?l) (box-at ?b ?l))
    :effect (and (carrying ?r ?b) (not (box-at ?b ?l))) )
  (:action drop
    :parameters (?r - robot ?b - box ?l - location)
    :precondition (and (at ?r ?l) (carrying ?r ?b))
    :effect (and (box-at ?b ?l) (not (carrying ?r ?b))) )
  (:action clean-nearby
    :parameters (?r - robot ?l - location)
    :precondition (at ?r ?l)
    :effect (forall (?x - location)
      (when (adjacent ?l ?x) (clean ?x))))
)
  \end{minted}

    В этом примере моделируется поведение роботов на складе:
    перемещение между локациями, управление коробками и очистка территорий.
    Используются возможности языка PDDL: \texttt{:strips}, \texttt{:typing}, \texttt{:quantified-preconditions}.
    Определены три типа объектов: роботы, коробки, локации.
    Предикаты задают положение роботов и коробок, факт переноски, чистоту и смежность локаций.
    Действия позволяют роботу: перемещается между соседними локациями,
    поднимать коробку из текущей локации, опускать коробку в текущей локации, очищать все смежные с текущей локации.

\item \textbf{Файл задачи} указывает,
  какие объекты существуют в данной конкретной задаче,
    каково начальное состояние и какое целевое состояние требуется достичь.
    Вот пример соответствующего файла задачи:

  \begin{minted}{lisp}
(define (problem mini-a)
  (:domain mini-warehouse)
  (:objects
    r1 - robot
    b1 b2 - box
    l1 l2 l3 - location)
  (:init
    (at r1 l1)
    (box-at b1 l2)
    (box-at b2 l3)
    (adjacent l1 l2)
    (adjacent l2 l3)
    (adjacent l1 l3))
  (:goal (and
    (box-at b1 l3)
    (clean l2)))
)
  \end{minted}

  Здесь задача использует домен \texttt{mini-warehouse}. 
  В задаче заданы: один робот, две коробки, три локации.
  Робот находится в \texttt{l1}, коробки \texttt{b1} и \texttt{b2} -- в \texttt{l2} и \texttt{l3} соответственно.
  Все локации попарно смежны.
  Целью является: коробка \texttt{b1} находится в локации \texttt{l3} и локация \texttt{l2} должна быть очищена.
\end{enumerate}

Такое разделение обеспечивает переиспользуемость:
один и тот же домен можно использовать для множества различных задач.
Кроме того, это упрощает тестирование и отладку планирующих систем.
Благодаря своей формальной структуре PDDL поддерживает автоматическую проверку корректности задач
и их совместимости с определением домена,
что делает его мощным инструментом для разработки и исследования алгоритмов планирования.

Решением для такой этой конкретной проблемы будет следующая последовательность действий:
\begin{minted}{lisp}
(
  (move r1 l1 l2)
  (pick-up r1 b1 l2)
  (move r1 l2 l3)
  (drop r1 b1 l3)
  (clean-nearby r1 l3)
)
\end{minted}

Задача планировщика --- принять на вход описание домена и задачи, и вывести последовательность действий для ее решения.

\section{Реализация планировщика}

\subsection{Выбор языка программирования}

В рамках данной работы в качестве основного языка программирования был выбран Rust~\cite{rust}.
Этот язык обладает рядом значимых преимуществ по сравнению с другими широко используемыми языками,
такими как C++, Python, Java и C\#,
что делает его особенно подходящим для разработки высоконадежных и производительных программных систем.
Наиболее важными характеристиками Rust, актуальными для реализации поставленных задач, являются следующие:

\begin{enumerate}
  \item \textbf{Высокая производительность:}
    \begin{itemize}
      \item \textit{Эффективность выполнения:} Rust компилируется в низкоуровневый машинный код, что обеспечивает производительность, сопоставимую с языками C и C++.
      \item \textit{Отсутствие виртуальной машины:} В отличие от языков, использующих виртуальные машины (например, Java), Rust исполняется напрямую, без дополнительного уровня абстракции, что способствует снижению накладных расходов и увеличению скорости выполнения.
    \end{itemize}

  \item \textbf{Безопасность работы с памятью:}
    \begin{itemize}
      \item \textit{Гарантированное управление памятью:} Система владения и заимствования Rust исключает возможность утечек памяти и освобождает разработчика от необходимости ручного управления ресурсами.
      \item \textit{Предотвращение типичных ошибок:} Механизмы компилятора предотвращают доступ к уже освобождённой памяти, использование недействительных указателей и другие распространённые ошибки управления памятью.
      \item \textit{Выявление ошибок на этапе компиляции:} Благодаря строгой системе проверки, большинство ошибок, связанных с управлением памятью, обнаруживаются до запуска программы, что повышает надёжность конечного решения.
    \end{itemize}

  \item \textbf{Развитая система типов и абстракций:}
    \begin{itemize}
      \item \textit{Статическая типизация:} Rust проводит строгую проверку типов на этапе компиляции, что способствует раннему обнаружению логических ошибок.
    \item \textit{Алгебраические типы данных:} Поддержка структур и перечислений (enums) позволяет выразительно и формально описывать сложные состояния предметной области.
    \item \textit{Система трейтов (traits):} Одним из ключевых элементов абстракции в Rust является механизм трейтов, который позволяет описывать обобщённое поведение для различных типов. Это приближает язык к концепции type classes из функционального программирования и обеспечивает мощную поддержку полиморфизма без потерь производительности.
    \end{itemize}
\end{enumerate}

Немаловажным также является тот факт, что из всех упомянутых языков программирования,
мы лучше всего знакомы с Rust, что только добавляет к уже упомянутым преимуществам.
Но при всех его преимуществах важно понимать также недостатки этого языка:

\begin{enumerate}
  \item Крутая кривая обучения: Понимание концепций владения, заимствования и жизненных циклов требует времени и усилий.
  \item Вербозность
    \begin{itemize}
      \item Жизненные циклы: Иногда требуется явное указание жизненных циклов, что может делать код более многословным и трудным для чтения.
      \item Типажи и обобщенные типы: Использование обобщенных типов и типажей может приводить к увеличению сложности кода и затруднять его чтение.
    \end{itemize}
  \item Жесткость системы типов: Система типов Rust может быть слишком строгой в некоторых случаях, требуя явного указания типов или дополнительных аннотаций там, где другие языки могут быть более гибкими.
\end{enumerate}

Вербозность и жесткость системы типов могут быть проблематичными,
особенно при написании данного документа.
Поэтому все примеры кода, присутствующие здесь,
не будут соответствовать реальной реализации программы,
они будут упрощены для передачи основной сути программы,
и будут порой игнорировать синтаксические и семантические правила,
чтобы не загромождать пример кода чрезвычайной вербозностью.

\subsection{Парсинг PDDL файла}

Первое, что наша программа должна сделать -- это прочесть два файла с описанием домена задачи
и конкретной проблемы в этом домене. Парсинг языка PDDL выходит за рамки этой задачи,
поэтому мы можем взять готовую реализацию в виде библиотеки pddl~\cite{pddl-crate}.
Эта библиотека парсит PDDL версии 3.1 основываясь на документации из Planning.Wiki~\cite{planning-wiki}.
Хотя эта библиотека поддерживает все необходимые элементы спецификации PDDL 3.1,
результаты парсинга очень сильно совпадают с BNF описанием языка из ~\cite{pddl-bnf}. 
Чтобы с ними можно было эффективно работать, нам необходимо преобразовать их в наши собственные типы.

\subsection{Основные компоненты}

\subsubsection{Объекты и Типы}

Система типов позволяет нам создавать базовые типы и подтипы,
к которым могут применяться предикаты.
Мы используем типы, чтобы ограничить объекты, которые могут выступать в качестве параметров действия.
Типы и подтипы позволяют нам задавать как общие, так и специфические действия и предикаты (листинг~\ref{code:entities}).
Блок для существующих типов мы можем создать объекты и константы (которые в этом планировщике реализованы одним и тем же образом).
Они описывают конкретные объекты, которые существуют в рамках нашей задачи.
Каждый объект должен иметь уникальное имя и быть типизирован.

\begin{figure}
  \begin{minted}{rust}
let mut entities = EntityStorage::default();

let t1 = entities.get_or_create_type("foo");
let t2 = entities.get_or_create_type("bar");
let _ = entities.create_inheritance(&t1, &t2);

let o1 = entities.get_or_create_object("o1", &t1);
let o2 = entities.get_or_create_object("o2", &t2);
  \end{minted}
  \captionof{listing}{Типы и объекты}
  \label{code:entities}
\end{figure}

\subsubsection{Предикаты}

Предикаты применяются к объектам определённого типа или ко всем объектам.
В любой момент выполнения плана предикаты либо истинны, либо ложны, и если они не заданы явно,
предполагается, что они ложны.
Структура \texttt{Predicate} (листинг~\ref{code:predicate}) реализующая предикаты PDDL параметризована типом,
реализующим \texttt{PredicateValue}, что позволяет использовать её как с конкретными,
так и с абстрактными значениями. Она предназначена для представления предикатов в определениях действий,
с возможностью выбора стратегии вычисления.
Также предусмотрен механизм различения логически независимых,
но синтаксически совпадающих предикатов, необходимый при преобразованиях логических выражений.

\begin{figure}
  \begin{minted}{rust}

enum EvalStrat {
    State,
    Equality,
}

struct Predicate<V: PredicateValue> {
    #[getset(get = "pub")]
    name: InternerSymbol,
    #[getset(get = "pub")]
    arguments: Vec<TypeHandle>,
    #[getset(get = "pub")]
    values: Vec<V>,
    #[getset(get = "pub")]
    unique_marker: Marker,
    eval_strat: EvalStrat,
}

type LiftedPredicate = Predicate<SymbolicValue>;
type GroundPredicate = Predicate<ObjectHandle>;
type ScopedPredicate = Predicate<ScopedValue>;
  \end{minted}
  \captionof{listing}{Предикат}
  \label{code:predicate}
\end{figure}

Трейт \texttt{PredicateValue} (листинг~\ref{code:valuetrait}) задаёт интерфейс для объектов,
которые могут выступать в роли значений предиката.
Таких объектов существует несколько, они позволяют ограничить допустимые значения, которые принимают предикаты во время решения задачи.
Например, \texttt{SymbolicValue} описывает все значения, которые предикат может принять при декларации действия,
а \texttt{ScopedValue} используется при описании целей, но также используется во время ее решения.

\begin{figure}
  \begin{minted}{rust}
trait PredicateValue: Debug + Clone + PartialEq + Ord {
    fn r#type(&self) -> TypeHandle;
}

enum SymbolicValue {
    Object(ObjectHandle),
    ActionParameter(ActionParameter),
    BoundVariable(BoundVariable),
}

enum ScopedValue {
    Object(ObjectHandle),
    BoundVariable(BoundVariable),
}

impl PredicateValue for SymbolicValue {
  // ...
}

impl PredicateValue for ScopedValue {
  // ...
}
  \end{minted}
  \captionof{listing}{Параметр предиката}
  \label{code:valuetrait}
\end{figure}

\subsubsection{Выражение}

Предикаты могут складываться в логические выражения логики первого порядка (листинг~\ref{code:foexpression}).
Выражения описываются данным типом рекурсивно и состоять из предикатов, логических операций и кванторов.
Однако из-за того, что это свободное выражение, работать с ним не удобно.
Поэтому на первом шаге программа преобразовывает такое выражение в предварительную нормальную форму (листинг~\ref{code:pdnf}).
В ней все кванторы находятся в начале, а оставшаяся часть (матрица) дополнительно преобразовывается в дизъюнктивную нормальную форму.
Выражение в такой форме эквивалентно изначальному.

\begin{figure}
  \begin{minted}{rust}
enum Quantifiers<E: FOExpression, P: IsPredicate> {
    ForAll(ForAll<E, P>),
    Exists(Exists<E, P>),
}

enum QuantifiedFormula<P: IsPredicate> {
    Quantifier(Quantifiers<QuantifiedFormula<P>, P>),
    And(Vec<QuantifiedFormula<P>>),
    Or(Vec<QuantifiedFormula<P>>),
    Not(Box<QuantifiedFormula<P>>),
    Pred(P),
}

impl FOExpression for QuantifiedFormula {
  // ..
}
  \end{minted}
  \captionof{listing}{Формула}
  \label{code:foexpression}
\end{figure}

\begin{figure}
  \begin{minted}{rust}
struct Pdnf<P: IsPredicate> {
    prefix: Vec<QuantifierSymbol>,
    matrix: Dnf<P>,
}

struct Dnf<P: IsPredicate> {
     clauses: BTreeSet<BTreeSet<Primitives<P>>>,
}
  \end{minted}
  \captionof{listing}{Предварительная дизъюнктивная нормальная форма}
  \label{code:pdnf}
\end{figure}

\subsubsection{Действия}

Действие определяет преобразование состояния мира.
Это преобразование, представляет собой действие,
которое может быть выполнено в ходе выполнения плана,
например, поднятие объекта, строительство чего-либо или другое изменение.
Действие состоит из трех частей (листинг~\ref{code:action}):
\begin{enumerate}
  \item Параметры, которые определяют объекты, к которым применяется действие, и, следовательно, предикаты, которые будут проверяться и изменяться в дальнейшем.
  \item Предусловие -- полное логическое выражение, которое должно быть выполнено, чтобы действие могло быть применено.
  \item Эффект состоит из конъюнктивного логического выражения,
    которое определяет, какие значения должны быть установлены в \textit{истину} или \textit{ложь} при выполнении действия (листинг~\ref{code:effect}).
    Хотя это логическое выражение, оно не обладает той же выразительной силой, что и предусловия, поскольку ему может быть сопоставлен только один набор истинностных значений.
    Это ограничивает нас в использовании операторов \texttt{and} и \texttt{not}, а так-же квантор \texttt{forall} для описания эффектов действия.
    Внутри эффектов могут также находится дополнительные условия, если используется операция \texttt{when}.
    В таком случае первый аргумент может быть полноценным выражением, потому что он используется только для проверки истинности,
    а не для изменения состояния.
\end{enumerate}

\begin{figure}
  \begin{minted}{rust}
struct Action {
    name: InternerSymbol,
    parameters: Vec<ActionParameter>,
    precondition: Pdnf<LiftedPredicate>,
    effect: ActionEffect,
}
  \end{minted}
  \captionof{listing}{Действие}
  \label{code:action}
\end{figure}

\begin{figure}
  \begin{minted}{rust}
enum ActionEffect {
    And(Vec<ActionEffect>),
    ForAll(ForAll<ActionEffect, LiftedPredicate>),
    When {
        condition: Pdnf<LiftedPredicate>,
        effect: Box<ActionEffect>,
    },
    Primitives(Primitives<LiftedPredicate>),
}
  \end{minted}
  \captionof{listing}{Эффект}
  \label{code:effect}
\end{figure}

\subsubsection{Описание задачи}

\begin{figure}
  \begin{minted}{rust}
struct Problem {
    name: InternerSymbol,
    domain_name: InternerSymbol,
    entities: EntityStorage,
    actions: NamedStorage<Action>,
    init: State,
    goal: QuantifiedFormula<ScopedPredicate>,
}

struct State {
    predicates: BTreeMap<PredicateKey, BTreeSet<GroundPredicate>>,
}
  \end{minted}
  \captionof{listing}{Проблема}
  \label{code:problem}
\end{figure}

В описании задачи, которую наша программа должна решить мы храним изначальное состояние,
цель, типы и объекты, а также набор всех возможных действий (листинг~\ref{code:problem}).
Само же состояние состоит из набора предикатов, которые верны для данного состояния
(если предикат отстутствует значит он неверен), набора констант, набора именнованных объектов
и всех определенных типов для объектов.

\subsection{Решение задачи}

Чтобы наша программа могла решить задачу планирования, необходимо,
чтобы существовал механизм перехода из одного состояния в другое посредством действий.
Для этого нужно:
\begin{enumerate}
  \item Найти действие, условие которого выполняется
  \item Определить конкретные объекты, с которыми можно применить эффекты действия
  \item Изменить состояние с помощью эффектов
\end{enumerate}

\subsubsection{Проверка условий действия}

Чтобы проверить, выполняется ли предусловие действия в заданном состоянии,
необходимо реализовать механизм, преобразующий \texttt{LiftedPredicate}
--- содержащие параметры действия, ограниченные типами,
а также связанные переменные в кванторах ---
в \texttt{GroundPredicate}, содержащие только конкретные объекты.
Для этого используется метод \texttt{State::ground\_action} (листинг~\ref{code:groundaction}).
В нём мы обрабатываем каждую дизъюнктивную клаузу предусловия,
приведённого к предикатной дизъюнктивной нормальной форме (PDNF), отдельно.
Это возможно благодаря тому, что наш алгоритм преобразования произвольного логического выражения в PDNF
производит полную дизъюнктивную нормальную форму, в которой все предикаты присутствуют в каждой клаузе.
Следовательно, все параметры действия также присутствуют в каждой клаузе.
Это избавляет нас от необходимости учитывать случаи их отсутствия.

Клауза содержит несколько предикатов, некоторые из которых могут быть отрицаны.
Для каждого положительного предиката мы можем просто проверить,
с какими объектами он присутствует в состоянии.
Это даёт отображение от параметра действия и связанной переменной (которая будет использоваться позже) к множеству объектов,
с которыми предикат истинно выполняется в этом состоянии (листинг~\ref{code:groundpredicate}).
С отрицательными предикатами задача сложнее: нам необходимо найти все объекты, с которыми вложенный положительный предикат \textit{не} выполняется.
Это достигается путём применения той же процедуры, что и для положительных предикатов,
а затем вычитанием полученного множества объектов из множества всех объектов, определённых в данной задаче.
Мы повторяем этот процесс для каждого предиката в клаузе, а затем объединяем полученные отображения (листинг~\ref{code:handleclause}).
Если один и тот же параметр встречается в нескольких предикатах,
мы пересекаем множества объектов, соответствующих этому параметру, что гарантирует выполнение всех предикатов одновременно.
Таким образом, мы обрабатываем матрицу PDNF.

На следующем этапе необходимо применить префикс,
чтобы дополнительно сузить множество возможных конкретизаций выражения (листинг~\ref{code:evalprefix}).
Префикс включает два квантора: \textit{forall} и \textit{exists}.
Изолированно для успешной проверки квантора \textit{forall} требуется,
чтобы выражение было истинно для всех объектов соответствующего типа, указанного при определении квантора.
Для \textit{exists} достаточно, чтобы выражение было истинно хотя бы для одного объекта.
Поскольку кванторы в префиксе могут появляться в произвольном порядке,
мы решаем их комбинацию методом полного перебора всех возможных перестановок связанных переменных.
Сначала мы строим отображение от связанной переменной к конкретному объекту,
а затем проверяем, существует ли такая комбинация в отображении, полученном на предыдущем шаге.
Если все комбинации проходят проверку, это означает, что соответствующая клауза делает всё исходное выражение истинным.

В конце алгоритма связанные переменные исключаются из результата, после чего все решения для отдельных клауз объединяются в одно множество.
Оно и представляет собой результат работы алгоритма,
предоставляя несколько <<параллельных>> способов преобразования обобщённого выражения в конкретизированное.

\begin{figure}
  \begin{minted}{rust}
pub fn ground_action<'a>(&'a self, action: &'a Action) -> BTreeSet<ParameterGrounding<'a>> {
    let num_params = action.parameters().len();
    let prefix = action.precondition().prefix();
    action
        .precondition()
        .matrix()
        .clauses
        .iter()
        .filter_map(|clause| self.handle_clause(clause))
        .filter(|r| {
            r.iter()
                .filter(|(k, v)| {
                    matches!(k, LiftedValue::ActionParameter(_, _)) && !v.is_empty()
                })
                .count()
                == num_params
        })
        .filter(|clause_map| eval_dnf_clause_with_quantifier_prefix(prefix, clause_map))
        .map(|map| {
            map.into_iter()
                .filter_map(|(k, v)| match k {
                    LiftedValue::ActionParameter(_, ap) => Some((ap, v)),
                    LiftedValue::BoundVariable(_, _) => None,
                })
                .collect::<BTreeMap<_, _>>()
        })
        .map(ParameterGrounding)
        .collect::<BTreeSet<_>>()
}
  \end{minted}
  \captionof{listing}{Конкретизация предусловия действия}
  \label{code:groundaction}
\end{figure}

\begin{figure}
  \begin{minted}{rust}
fn handle_clause<'a>(
    &'a self,
    and: &'a BTreeSet<Primitives<LiftedPredicate>>,
) -> Option<BTreeMap<LiftedValue<'a>, BTreeSet<ObjectHandle>>> {
    let o = and
        .iter()
        .fold(BTreeMap::default(), |mut acc: BTreeMap<_, _>, p| {
            self.handle_primitives(p).into_iter().for_each(|(k, v)| {
                acc.entry(k)
                    .and_modify(|s: &mut BTreeSet<_>| {
                        *s = s.intersection(&v).cloned().collect()
                    })
                    .or_insert_with(|| v.clone());
            });
            acc
        });
    if o.is_empty() { None } else { Some(o) }
}
  \end{minted}
  \captionof{listing}{Обработка отдельной клаузы ДНФ}
  \label{code:handleclause}
\end{figure}


\begin{figure}
  \begin{minted}{rust}
fn ground_predicate<'a>(
    &'a self,
    predicate: &'a LiftedPredicate,
) -> BTreeMap<LiftedValue<'a>, BTreeSet<ObjectHandle>> {
    let keys = predicate.lifted_values();
    let acc = || {
        keys.iter()
            .map(|&k| (k, BTreeSet::new()))
            .collect::<BTreeMap<_, _>>()
    };
    self.predicates
        .get(&predicate.into())
        .map(|preds| {
            preds
                .iter()
                .filter(|rp| rp.is_ground_form(predicate))
                .flat_map(|rp| {
                    keys.iter()
                        .zip(keys.iter().map(|k| rp.values()[k.idx()].dupe()))
                        .collect_vec()
                })
                .fold(acc(), |mut acc, (lv, v)| {
                    acc.entry(*lv)
                        .and_modify(|s: &mut BTreeSet<_>| {
                            let _ = s.insert(v.dupe());
                        })
                        .or_insert_with(|| BTreeSet::from([v]));
                    acc
                })
        })
        .unwrap_or_else(acc)
}
  \end{minted}
  \captionof{listing}{Конкретизация предиката}
  \label{code:groundpredicate}
\end{figure}


\begin{figure}
  \begin{minted}{rust}
pub fn eval_dnf_clause_with_quantifier_prefix(
    prefix: &[QuantifierSymbol],
    clause_map: &BTreeMap<LiftedValue<'_>, BTreeSet<ObjectHandle>>,
) -> bool {
    fn eval(
        prefix: &[QuantifierSymbol],
        clause_map: &BTreeMap<LiftedValue<'_>, BTreeSet<ObjectHandle>>,
        var_assignment: &mut BTreeMap<BoundVariable, ObjectHandle>,
    ) -> bool {
        let Some((head, tail)) = prefix.split_first() else {
            return clause_map.iter().all(|(k, v)| match k {
                LiftedValue::ActionParameter(_, _) => true,
                LiftedValue::BoundVariable(_, bv) => v.contains(
                    var_assignment
                        .get(bv)
                        .expect("BoundVariable is being used without a quantifier."),
                ),
            });
        };

        match head {
            QuantifierSymbol::ForAll(variables) => variables
                .iter()
                .map(|v| v.r#type.get_objects().into_iter().map(|e| (v.dupe(), e)))
                .multi_cartesian_product()
                .all(|v| with_var_assignment(var_assignment, v, |x| eval(tail, clause_map, x))),
            QuantifierSymbol::Exists(variables) => variables
                .iter()
                .map(|v| v.r#type.get_objects().into_iter().map(|e| (v.dupe(), e)))
                .multi_cartesian_product()
                .any(|v| with_var_assignment(var_assignment, v, |x| eval(tail, clause_map, x))),
        }
    }
    eval(prefix, clause_map, &mut BTreeMap::new())
}
  \end{minted}
  \captionof{listing}{Проверка префикса}
  \label{code:evalprefix}
\end{figure}

\subsubsection{Применение эффектов}

Метод \texttt{Action::ground\_effect} (листинг~\ref{code:groundeffect}) принимает на вход возможные варианты конкретизации предусловия,
полученные на предыдущем этапе, и применяет соответствие, чтобы получить несколько версий эффекта действия.
Внутри него для каждого <<параллельного>> решения осуществляется обход рекурсивной структуры \texttt{ActionEffect}.
Когда достигается \texttt{LiftedPredicate}, он преобразуется в набор \texttt{GroundPredicate} ---
по всем комбинациям параметров действия и связанных переменных, присутствующих в предикате.
Затем предикаты оборачиваются в перечисление \texttt{ModifyState}, которое лучше передаёт наше намерение изменить состояние.
Если в эффекте присутствует условие \textit{when},
более сложное логическое выражение преобразуется в его конкретную версию и затем оценивается в текущем состоянии,
чтобы определить, следует ли применять соответствующий эффект.
Если да --- алгоритм переходит глубже в выражение \textit{when},
если нет --- соответствующая ветка рекурсии завершает выполнение.
В результате алгоритма формируется множество множеств \texttt{ModifyState}.
С помощью эвристики можно выбрать одно из них,
если необходимо применить это действие среди нескольких,
предусловия которых выполняются в текущем состоянии.

\begin{figure}
  \begin{minted}{rust}
pub fn ground_effect<'a>(
    &'a self,
    parameter_groundings: &'a BTreeSet<ParameterGrounding<'a>>,
    state: &impl EvaluationContext<GroundPredicate>,
) -> Result<BTreeSet<BTreeSet<ModifyState>>, PredicateError> {
    fn ground_effect_flat(
        effect: &ActionEffect,
        grounding: &BTreeMap<&ActionParameter, ObjectHandle>,
        state: &impl EvaluationContext<GroundPredicate>,
    ) -> Result<BTreeSet<ModifyState>, PredicateError> {
        match effect {
            ActionEffect::And(effects) => effects
                .iter()
                .map(|e| ground_effect_flat(e, grounding, state))
                .try_fold(BTreeSet::new(), |mut acc, el| {
                    el.map(|e| {
                        acc.extend(e);
                        acc
                    })
                }),
            ActionEffect::ForAll(forall) => {
                ground_effect_flat(forall.expression(), grounding, state)
            }
            ActionEffect::When { condition, effect } => {
                condition.into_scoped(grounding).and_then(|condition| {
                    if condition.eval(state) {
                        ground_effect_flat(effect, grounding, state)
                    } else {
                        Ok(BTreeSet::new())
                    }
                })
            }
            ActionEffect::Primitives(p) => match p {
                Primitives::Pred(pred) => pred
                    .as_ground(grounding)
                    .map(|r| r.into_iter().map(ModifyState::Add).collect::<BTreeSet<_>>()),
                Primitives::Not(not) => not
                    .as_ground(grounding)
                    .map(|r| r.into_iter().map(ModifyState::Del).collect::<BTreeSet<_>>()),
            },
        }
    }

    parameter_groundings
        .iter()
        .flat_map(|grounding| {
            grounding
                .as_simple()
                .map(|grounding| ground_effect_flat(&self.effect, &grounding, state))
        })
        .collect::<Result<BTreeSet<_>, _>>()
}
  \end{minted}
  \captionof{listing}{Конкретизация эффекта}
  \label{code:groundeffect}
\end{figure}

\subsection{Решение задачи}

Для решения задач планирования, определённых в терминах типизированных обобщенных представлений,
реализован вариант алгоритма поиска A*, адаптированный для работы с конкретизацией параметризованных действий в контексте текущего состояния.
Каждый узел поиска представлен структурой \texttt{Node} (листинг~\ref{code:node}) содержащей в себе:

\begin{itemize}
    \item \texttt{state} --- текущее конкретизированное состояние (набор известных предикатов).
    \item \texttt{cost} --- накопленная стоимость (например, длина пути) до данного состояния.
    \item \texttt{estimated\_total} --- сумма фактической стоимости и эвристической оценки ($f = g + h$).
    \item \texttt{path} --- последовательность действий и соответствующих подстановок параметров, приведшая к данному состоянию.
\end{itemize}
Очередь поиска поддерживается в виде приоритетной очереди, упорядоченной по значению \texttt{estimated\_total}.

\begin{figure}
  \begin{minted}{rust}
struct Node {
    state: State,
    cost: usize,
    estimated_total: usize,
    path: Vec<(InternerSymbol, ParameterGrounding)>,
}
  \end{minted}
  \captionof{listing}{Узел}
  \label{code:node}
\end{figure}


Основу алгоритма составляет функция \texttt{astar\_search} (листинг~\ref{code:search}).
Функция принимает на вход объект \texttt{Problem} и эвристику для поиска.
Функция осуществляет поиск от начального состояния до состояния, удовлетворяющего цели, через конкретизацию и применение действий.

Алгоритм поддерживает следующие структуры:
\begin{itemize}
    \item Приоритетную очередь (\texttt{open}) с узлами, упорядоченными по предполагаемой общей стоимости.
    \item Множество посещённых состояний (\texttt{visited}).
    \item Отображение стоимости (\texttt{cost\_so\_far}), хранящее наименьшую известную стоимость достижения каждого состояния.
\end{itemize}

На каждой итерации извлекается наиболее перспективный узел. Если цель выполнена (\texttt{problem.goal.eval(state)} возвращает \texttt{true}), то возвращается накопленный путь.

В противном случае:
\begin{enumerate}
    \item Перебираются все действия из домена.
    \item Для каждого действия выполняется конкретизация параметров в контексте текущего состояния с помощью \texttt{state.ground\_action(action)}.
    \item Рассчитываются эффекты конкретизированного действия с помощью \\
      \texttt{action.ground\_effect(...)} и применяются к состоянию через \\
      \texttt{state.apply\_modification(...)}.
    \item Новые состояния добавляются в очередь, если они ещё не были посещены.
\end{enumerate}

\begin{figure}
  \begin{minted}{rust}
pub fn astar_search<H>(problem: &Problem, heuristic: H)
  -> Option<Vec<(InternerSymbol, ParameterGrounding)>> 
where
    H: Fn(&State) -> usize
{
    // ..
}
  \end{minted}
  \captionof{listing}{Функция поиска}
  \label{code:search}
\end{figure}

\begin{verbatim}
\end{verbatim}

\subsubsection{Эвристическая функция}

В текущей реализации используется заглушка для эвристики (листинг~\ref{code:heuristic}).
Таким образом, поиск A* эквивалентен поиску с равной стоимостью (алгоритм Дейкстры).
В перспективе эвристическая функция может быть дополнена доменно-специфическими оценками,
такими как количество неудовлетворённых подцелей или использование расслабленного графа планирования.

\begin{figure}
  \begin{minted}{rust}
fn heuristic(_state: &State) -> usize {
    0
}
  \end{minted}
  \captionof{listing}{Эвристика}
  \label{code:heuristic}
\end{figure}

\subsubsection{Интеграция с планировщиком}

Поиск осуществляется над объектом \texttt{Problem} (листинг~\ref{code:problem}).

Алгоритм предполагает, что планировщик поддерживает:
\begin{itemize}
    \item Управление сущностями и типами через \texttt{EntityStorage},
    \item Конкретизацию и применение действий через \texttt{Action::ground\_action}\\
      и \texttt{Action::ground\_effect},
    \item Переход между состояниями с помощью \texttt{State::apply\_modification}.
\end{itemize}

Результатом выполнения поиска является план в виде последовательности пар \texttt{(действие, подстановка параметров)},
который переводит начальное состояние в состояние, удовлетворяющее целевой формуле.

\section{Дальнейшая работа}

В текущей реализации планировщика использован алгоритм A* для поиска решения задач планирования, описанных в формате PDDL.
Несмотря на преимущества Rust в области управления памятью и безопасности, производительность нашего планировщика оставляет желать лучшего.
Основной причиной является субоптимальность подхода, связанная с использованием перебора для нахождения соответствующих значений для переменных в действиях.

Большим недостатком текущего подхода является перебор всех значений для валидации префикса предусловия.
Это приемлимо в большинстве задач, так как кванторы встречаются достаточно редко, но это может быть проблемным в дальнейшем.
Одним из способов оптимизации может быть удаление кванторов \textit{exists} из префикса,
либо путем сколемизации, либо раскрытием их в выражение \textit{or}, что может значительно увеличить размер выражения.

Внедрение более сложных эвристических функций, основанных на релаксации задачи~\cite{stolba2014relaxation},
может существенно улучшить направленность поиска и уменьшить количество исследуемых состояний.
А использование методов, таких как эвристика \texttt{hmax} или \texttt{hadd},
которые учитывают минимальное количество шагов до достижения цели, может повысить эффективность поиска.

\section{Заключение}

В данной работе была представлена реализация планировщика,
использующего язык Rust и основанного на формализме PDDL.
Мы рассмотрели основные аспекты и компоненты планировщиков,
включая начальные состояния, целевые состояния, действия и их предусловия и эффекты.
Также была предоставлена детальная имплементация алгоритма A* для поиска решений задач планирования.

Реализация планировщика на Rust показала хорошие базовые возможности 
и продемонстрировала потенциал для дальнейшего развития.
Применение предложенных улучшений позволит значительно повысить производительность
и применимость планировщика для решения более сложных задач планирования.
В будущей работе мы будем стремиться к реализации этих улучшений
и проведению тестов на сложных и разнообразных задачах,
что позволит оценить эффективность предложенных методов и подходов.

\newpage

\section{Список литературы}
\printbibliography [heading=none]

\end{document}
