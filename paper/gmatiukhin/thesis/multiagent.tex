\chapter{Мультиагентный планировщик}

Предыдущая глава была посвещена разработке и функционированию одноагентного планировщика,
который составляет основу для мультиагентного планировщика, представленного в этой главе.
Одноагентный планировщик, сосредоточенный на формулировке целей,
последовательности действий и эффективном исследовании осуществимых планов,
предоставляет базовые механизмы, которые адаптируются и расширяются в мультиагентной среде.

Однако с переходом к планированию с участием нескольких автономных агентов
сложность задачи существенно возрастает.
Каждому агенту необходимо не только решать свои индивидуальные цели,
но также координировать действия, обмениваться информацией
и, при необходимости, корректировать свои действия в ответ на планы других.
В данной главе представлен мультиагентный планировщик,
основанный на архитектуре одноагентного планировщика
и дополненный механизмами координации,
приоритезации целей и обмена информацией.

Мультиагентный планировщик предполагает,
что все агенты обладают общими знаниями о предметной области,
что позволяет им опираться на единое представление
о среде и задачах. Каждый агент осуществляет планирование независимо,
при этом обмениваясь планами и обновлениями с другими агентами
посредством распределённого коммуникационного протокола.
Такой децентрализованный процесс планирования обеспечивает гибкость,
устойчивость и масштабируемость, делая систему подходящей для сложных
реальных сценариев, таких как рои роботов или распределённые логистические сети.

\section{Модель планирования и архитектура системы}

Модель планирования следует формализму, подобному STRIPS~\cite{FIKES1971189}:
типизированные объекты, детерминированные действия,
пропозициональные описания состояний
и классическое достижение целей,
унаследованные от одноагентного планировщика.
Все агенты наблюдают одно и то же начальное состояние и спецификацию домена,
но получают различные множества целей.
Ключевым моментом является то, что агенты реализованы как независимые сущности,
каждая из которых оснащена собственной экземпляром
планировщика, соответствующего API на языке Rust, основанному на типажах.

В частности, система планирования построена вокруг
типажов \cd{Solver} и \cd{Heuristic}, представленных в предыдущей главе.
Они определяют модульный и расширяемый интерфейс
для реализации новых алгоритмов поиска и эвристических стратегий.
Оба типажа включают метод \cd{can\_solve()},
который проверяет, способен ли данный планировщик обрабатывать
необходимые особенности домена (такие как условные эффекты или кванторные выражения).
Типаж \cd{Heuristic} дополнительно определяет метод \cd{prepare()}
для предварительной обработки таких структур,
как \jar{Граф планирования с упрощением}
(Relaxed Plannig Graph, RPG)~\cite{BLUM1997281,Hoffmann_2001,bryce2007planninggraph},
и метод \cd{call()} для оценки конкретного состояния.
Типаж \cd{Solver} инкапсулирует основную логику планирования
через метод \cd{solve()}, который принимает \cd{Problem} и \cd{Heuristic}
и возвращает \cd{Plan}, содержащий список \cd{GroundActions},
каждое из которых определяет имя действия и конкретный набор привязанных объектов.
Такая абстракция позволяет разработчикам гибко заменять и комбинировать
решатели и эвристики, сохраняя при этом единый интерфейс.

\section{Достижимость целей и приоритезация}

Перед началом планирования каждый агент выполняет эвристический анализ достижимости
путём построения RPG из общего начального состояния.
Этот граф распространяет положительные эффекты действий, игнорируя эффекты удаления,
тем самым аппроксимируя, какие предикаты
--- а следовательно, и какие цели --- могут быть достижимы.
Цели, отсутствующие в множестве достижимых по релаксации, помечаются как потенциально недостижимые.
Однако данный процесс носит консервативный характер и не гарантирует
выявление всех недостижимых целей из-за своих упрощений.
Поэтому агенты не отбрасывают такие цели полностью, а лишь понижают их приоритет в планировании.

Для управления конкурирующими задачами и ограниченными ресурсами планирования
в планировщик интегрирована приоритезация целей.
Каждый агент назначает приоритеты целям на основе нескольких критериев:
оценочной стоимости достижения (вычисляемой по RPG),
собственных предпочтений или функции полезности,
а также количества других агентов, разделяющих данную цель.
Последний критерий способствует кооперативному поведению
и помогает разрешать неоднозначности в частично пересекающихся множествах целей.
Важно отметить, что приоритезация целей используется исключительно как эвристика поиска;
она влияет на порядок планирования, но не нарушает \jar{корректность},
которая здесь определяется строго как сохранение корректности плана
в отношении достижения целей и логической непротиворечивости.

\section{Планирование через уточнение}

Планирование осуществляется с использованием
\jar{стратегии уточнения с частичным порядком}
(Partial Order Planning)~\cite{Weld_1994,BENTON2009562}.
Изначально создаётся каркас плана, состоящий из двух абстрактных шагов:
\pc{Start}, представляющего начальное состояние,
и \pc{End}, кодирующего приоритезированное множество целей агента.
Агент разрешает открытые предусловия \pc{End} в порядке убывания приоритета целей.
Для разрешения условия он либо повторно использует действие,
уже присутствующее в плане (возможно, добавленное другим агентом),
либо вводит новые действия, чьи эффекты удовлетворяют данному условию,
вставляя соответствующие \jar{каузальные связи} и \jar{временные ограничения} в план.
Этот процесс, в свою очередь, может породить новые открытые условия,
которые уточняются рекурсивно аналогичным образом.

Важно отметить, что поскольку
одноагентный планировщик не поддерживает явные временные конструкции
(такие как длительности, временные окна или метрическое время),
их также не поддерживает и мультиагентная версия.
Термин \jar{временное ограничение}
не относится к реальному времени или действиям с временными метками.
Вместо этого он обозначает \jar{ограничения порядка} между действиями ---
то есть логическое требование, чтобы одно действие произошло раньше другого.
Однако, если в дальнейшем будут реализованы временные конструкции,
то помимо порядка действий будут учитываться и интервалы времени,
которые должны учитываться действиями.

Процесс уточнения направляется эвристикой,
а неполные подпланы оцениваются динамически
с использованием выбранной реализации \cd{Heuristic}.
Уточнение продолжается итеративно до тех пор,
пока не будет найден согласованный план,
удовлетворяющий достижимому подмножеству целей,
или пока не исчерпаются возможные шаги уточнения.

Если некоторые цели остаются неразрешёнными
--- будь то из-за некорректной оценки достижимости или их действительной недостижимости ---
агент продолжает работу с частичным планом,
охватывающим наиболее ценные цели, которые он может достичь.
Эти решения принимаются локально,
но агенты могут возвращаться к неразрешённым целям на последующих этапах планирования,
потенциально делегируя их другим агентам или повторяя попытку с изменёнными допущениями.

\section{Межагентная коммуникация}

После построения плана агент передаёт его результат другим участникам.
Хотя подробная реализация протокола связи выходит за рамки данной работы,
предполагается, что передаваемая информация следует структурированному формату,
разработанному для инкапсуляции ключевых выходных данных процесса планирования каждого агента.
Каждое сообщение включает следующие компоненты:
\begin{itemize}
  \item Идентификатор агента: Уникальный идентификатор, обозначающий
    источник сообщения. Это позволяет получателям ассоциировать полученные планы и решения
    с конкретными участниками сети.
  \item Достигнутые цели: Список предикатов, представляющих
    подмножество целей, назначенных агенту,
    которые были успешно достигнуты в построенном плане.
    Обычно это \cd{GroundPredicates},
    то есть конкретные инстанцированные формы абстрактных целевых условий.
  \item План действий: Последовательность \cd{GroundActions},
    каждая из которых включает имя действия и
    список полностью инстанцированных аргументов-объектов.
    Эта последовательность отражает шаги, которые агент намерен выполнить
    для достижения своих целей, упорядоченные в соответствии с ограничениями,
    полученными в процессе частичного планирования.
  \item Оценка плана (опционально): Числовая оценка,
    отражающая качество или полезность плана.
    Она может быть получена на основе эвристической оценки,
    стоимости планирования, эффективности по времени
    или некоторой предметно-ориентированной функции полезности.
    Включение такой оценки позволяет другим агентам принимать обоснованные решения
    в случаях, когда несколько агентов стремятся к перекрывающимся целям.
  \item Метаданные: Гибкий набор пар ключ-значение,
    используемый для передачи дополнительной информации.
    Это могут быть временные метки, эвристические метрики, весовые коэффициенты приоритетов,
    допущения, сделанные при планировании, или подсказки для координации.
    Поле метаданных поддерживает расширяемость
    без жёсткой привязки к структуре основного сообщения.
\end{itemize}

Такая структура сообщений обеспечивает возможность агентов
обмениваться планами в формате, который одновременно семантически насыщен
и синтаксически единообразен.
Включение достигнутых целей и оценок облегчает последующую координацию,
а метаданные поддерживают дополнительные сценарии, такие как
логгирование, отладка или асинхронная доработка планов.

Хотя данная коммуникация способствует взаимной осведомлённости и неформальному взаимодействию,
архитектура системы обеспечивает доминирование локальных решений агентов.
То есть каждый агент сохраняет полную автономию
в отношении своего процесса планирования и поведения.
Полученные планы или метаданные от других агентов трактуются как рекомендации;
они могут повлиять на приоритеты в планировании или инициировать перепланирование,
но никогда не накладывают прямых ограничений.
Такое проектное решение поддерживает асинхронную работу,
масштабируемость и устойчивость в условиях, когда агенты могут иметь различные роли,
прерывистую связь или расходящиеся цели.
Оно также позволяет агентам эффективно функционировать
даже при наличии устаревшей информации,
задержек в сообщениях или выхода из строя других участников.

\section{Обработка недостижимых целей}

Одной из отличительных черт данного мультиагентного планировщика
является его способность элегантно обрабатывать недостижимые цели и частичный успех плана.
В отличие от традиционных одноагентных планировщиков
(и несмотря на то, что он построен поверх такового),
которые либо находят полное решение, либо полностью терпят неудачу,
эта система специально разработана для эффективной работы в условиях частичной разрешимости,
неопределённости или динамической валидности целей.

\subsection{Устойчивость через частичные планы}

В процессе уточнения агенты могут определить,
что одна или несколько их целей не могут быть достигнуты ---
либо потому, что они недостижимы в текущем состоянии мира,
либо потому, что они конфликтуют с более приоритетными целями,
либо из-за необходимости координации с агентами,
которые ещё не внесли соответствующий вклад.
Вместо того чтобы прекращать процесс планирования в таких случаях,
каждый агент продолжает построение частичного плана,
достигающего наиболее ценных целей из своего множества,
определённых на основе эвристик приоритета и оценки достижимости.
Такое поведение не только преднамеренно,
но и критически важно для приложений в реальном мире,
где полное достижение целей часто невозможно
из-за ограничений, таких как время, энергопотребление, отказ оборудования
или непредсказуемые внешние события.
Относясь к частичным планам как к полноценным результатам,
система может поддерживать \jar{инкрементальное планирование}~\cite{BENTON2009562} и поэтапное выполнение,
с возможностью последующего пересмотра, делегирования
или оппортунистического завершения ранее нерешённых целей.

\subsection{Локальная автономия в условиях отказа}

Когда происходят отказы --- будь то из-за недостижимых целей,
неудачного выполнения действия или устаревших допущений ---
планировщик придерживается децентрализованного принципа:
локальные решения агентов доминируют.
Каждый агент независимо оценивает последствия отказа
на основе своего текущего плана, приоритетов и локального состояния.
Это означает, что агент может выбрать следующее:
\begin{itemize}
  \item Перераспределить приоритеты и перепланировать оставшиеся цели;
  \item Сообщить другим агентам о неразрешённых целях;
  \item Повторить попытку с альтернативными стратегиями или другими эвристиками;
  \item Отложить выполнение до появления дополнительной информации.
\end{itemize}

Поскольку никакой центральный координатор не управляет поведением агентов,
такая автономия обеспечивает устойчивость к частичным отказам:
агент, не сумевший построить план, не мешает другим продолжать работу,
а агент, потерпевший частичный отказ, всё равно может внести значимый вклад
в выполнение глобального множества задач.
Более того, неудачные планы или неразрешённые цели
могут быть пересмотрены оппортунистически ---
по мере появления новых действий или изменения состояния мира
в результате действий других агентов.
