\chapter*{Заключение}
\addcontentsline{toc}{chapter}{Заключение}

В данной работе была разработан мультиагентный планировщик,
ориентированный на координацию действий агентов с частично совпадающими
или конфликтующими целями. В работе реализован одноагентный планировщик
с поддержкой логики первого порядка, кванторов и дизъюнктивной нормальной формы,
что позволило эффективно обрабатывать сложные условия применимости действий.
Этот планировщик послужил основой для мультиагентной системы,
в которой агенты строят планы автономно,
но координируют действия через обмен информацией о состоянии среды и намерениях.
В результате каждый агент строит индивидуально рациональные планы,
которые согласуются с коллективной стратегией выполнения задач,
без необходимости централизованного управления.

В ходе систематического обзора литературы были выявлены основные подходы
к кооперации и разрешению конфликтов в мультиагентных системах,
а также отмечены их ограничения —
в том числе в части масштабируемости, прозрачности
и устойчивости к недостижимым целям.
Разработанная архитектура учитывает эти ограничения:
она масштабируется за счёт локального планирования,
устойчива за счёт частичных планов,
и поддерживает расширение за счёт модульного API,
написанного на языке Rust.
Таким образом, полученный фреймворк применим
в широком диапазоне задач --- от ройной робототехники до распределённой логистики.

\section*{Дальнейшая работа}

В текущей реализации агенты планируют свои действия
на основе заранее заданного описания среды и целей.
В дальнейшем планируется добавить поддержку динамического обновления задач,
включая возможность модификации целей во время исполнения,
что позволит моделировать более реалистичные сценарии.

Также в системе отсутствует поддержка жёстких ограничений
(например, ограничений по ресурсам или временных ограничений).
Добавление таких возможностей потребует расширения модели состояний
и модификации планировщика для обработки конфликтов на уровне ограничений.

Сейчас межагентное взаимодействие основано на обмене планами и эвристиками.
Перспективным направлением является внедрение аргументационных моделей:
агенты смогут не только делиться планами, но и убеждать друг друга
в выборе тех или иных действий,
опираясь на приоритеты и контекст.

Кроме того, перспективным направлением является разработка
и интеграция более выразительных эвристик.
В настоящее время используются эвристики на основе релаксации удалений эффектов,
такие как $h_{add}$ и $h_{max}$,
однако они склонны недооценивать стоимость действий
в насыщенных доменах с отрицательными и дизъюнктивными предусловиями.
Интеграция эвристик, основанных на обязательных действиях и фактах,
например LMCut~\cite{lmcut},
позволит повысить точность оценки стоимости
и улучшить эффективность планирования в сложных средах.
Возможна также разработка смешанных эвристик,
которые комбинируют разные подходы в зависимости от структуры задачи.

Наконец, возможным направлением развития является интеграция обучения:
эвристики и стратегии приоритезации целей могут быть получены
не вручную, а в результате обучения на симулированных эпизодах планирования.
Это позволит агентам адаптироваться к неизвестным или изменяющимся условиям среды.

Таким образом, представленная работа закладывает основу
для создания масштабируемых, гибких и адаптивных систем планирования,
способных работать в условиях реального мира.
