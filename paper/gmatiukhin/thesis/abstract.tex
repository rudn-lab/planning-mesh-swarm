\setstretch{1.2}
\begin{center}
  \textbf{Федеральное государственное автономное образовательное учреждение высшего образования} \\
  \textbf{<<Российский университет дружбы народов}\\
  \textbf{имени Патриса Лумумбы>>}\\

  \large \textbf{АННОТАЦИЯ}\\
  \normalsize \textbf{выпускной квалификационной работы} \\

  \underline{Матюхина Григория Васильевича}
\end{center}

на тему: \uline{Мультиагентное планирование в условиях несогласованности целей}

В данной работе рассматривается проблема мультиагентного планирования
в условиях несогласованности целей между агентами.
Основное внимание уделяется разработке и реализации планировщика,
способного координировать действия автономных агентов,
обладающих индивидуальными наборами целей
и функционирующих в общем пространстве состояний.
В работе проведён систематический обзор
существующих методов разрешения конфликтов и кооперации
в мультиагентных системах с использованием методологии PRISMA-ScR,
что позволило выявить ограничения текущих подходов
и сформулировать требования к архитектуре собственной системы.
Разработанный планировщик реализован на языке Rust
и способен функционировать на встраиваемых устройствах.
Его архитектура включает поддержку эвристического анализа достижимости целей,
механизмов приоритезации, обмена планами и устойчивого поведения
в условиях частичной недостижимости целей.
Результаты демонстрируют применимость предложенного решения в распределённых сценариях,
таких как рои роботов и логистические сети,
что подтверждает актуальность и практическую значимость работы.

\vspace*{\fill}

\noindent \begin{tabular}{p{0.33\linewidth} p{0.33\linewidth} p{0.33\linewidth}}
Автор ВКР & \underline{\phantom{signature sign}} & \underline{\phantom{Матюхин Григорий}} \\
& (Подпись) & (ФИО)
\end{tabular}

\thispagestyle{empty} 
