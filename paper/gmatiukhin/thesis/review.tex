\chapter{Анализ существующих методик}

В этом обзоре систематически анализируются методологии и исследовательские пробелы
с использованием PRISMA-ScR~\cite{prisma-src}, с акцентом на моделирование предметной
области, разрешение конфликтов и механизмы кооперации. Цель --- предоставить основу
для будущих разработок MAS, способствующих их более широкому применению в реальных
сценариях.

Ключевые цели исследования:

\begin{enumerate}
  \item Определение распространенных подходов к моделированию MAS.
  \item Анализ стратегий разрешения конфликтов.
  \item Исследование механизмов кооперации.
  \item Оценка эффективности существующих методологий.
\end{enumerate}

\section{Методы} 

\subsection{Протокол и регистрация}

Настоящий обзор следует структуре PRISMA-ScR~\cite{prisma-src},
обеспечивая систематичность и прозрачность изложения.
Перед сбором данных были определены исследовательские цели,
критерии включения и стратегии поиска.
Протокол не был зарегистрирован в PROSPERO,
так как скопинговые обзоры не входят в его сферу.\footnote{С протоколом можно ознакомится по ссылке 
\url{https://gmatiukhin.site/articles/MAS-Conflict-Solving-Review/}}

\subsection{Критерии отбора}

Данный обзор основан на структуре \jar{Популяция, Концепция, Контекст}
(Population, Concept, Context, PCC)~\cite{afc61c6cf471416489e36a4bc382d3b9}.
В него включены исследования, посвященные методам кооперации в системах с несколькими автономными агентами.
Под <<автономным агентом>> понимается интеллектуальный объект,
способный воспринимать окружающую среду,
принимать решения и действовать самостоятельно в рамках вычислительных и коммуникационных ограничений.
Это определение не ограничивается только роботами или беспилотными транспортными средствами.
Обзор включает как качественные, 
так и количественные эмпирические исследования,
исключая комментарии, аналитические заметки и тезисы конференций.

\begin{table}
  \centering
  \begin{footnotesize}
  \caption{Критерии включения/исключения}
  \label{tab:criteria}
  \begin{longtable}{|p{0.15\textwidth}|p{0.2\textwidth}|p{0.2\textwidth}|p{0.35\textwidth}|}
    \hline
    Критерий & Включение & Исключение & Обоснование\\
    \hline
    \hline
 Популяция        & Системы с несколькими автономными агентами, работающими как в реальных (роевые роботы, системы доставки), так и в виртуальных или симулированных средах. & Системы с одним агентом.                                                              & В обзоре рассматриваются кооперация и разрешение конфликтов, что неактуально для систем с одним агентом.                  \\
    \hline
 Концепция        & Алгоритмы, обеспечивающие кооперацию между автономными агентами в MAS.                                                                                   & Алгоритмы, ориентированные только на одиночные агенты.                                & Обзор сосредоточен на алгоритмах, способствующих кооперации в MAS.                                                        \\
    \hline
 Контекст         & Исследования, посвященные разработке алгоритмов кооперативного планирования и разрешения конфликтов в MAS с элементами искусственного интеллекта.        & Исследования, касающиеся разрешения конфликтов вне области искусственного интеллекта. & Обзор ориентирован исключительно на алгоритмы в области искусственного интеллекта, остальные области не рассматриваются.  \\
    \hline
 Язык             & Английский, русский, немецкий.                                                                                                                           & Другие языки.                                                                         & Рецензенты владеют только английским, русским и немецким языками и не имеют ресурсов для перевода статей с других языков. \\
    \hline
 Типы источников  & Рецензируемые научные статьи, эмпирические исследования.                                                                                                 & Нерецензируемые статьи, комментарии, письма в редакцию.                               & Использование рецензируемых источников гарантирует надежность и научную обоснованность представленных данных.             \\
    \hline
 Географический   & Любые страны.                                                                                                                                            & Нет исключений.                                                                       & Исследование кооперации и разрешения конфликтов в MAS не зависит от страны происхождения.                                 \\
    \hline
 Временной период & Любой.                                                                                                                                                   & Нет исключений.                                                                       & Область исследования достаточно узкая, поэтому решено включить все доступные научные работы.                              \\
    \hline
  \end{longtable}
  \end{footnotesize}
\end{table}

\subsection{Источники информации}

Поиск литературы проводился преимущественно в ScienceDirect,
выбранном из-за его обширного репозитория рецензируемых статей
по разрешению конфликтов и кооперации в MAS.
Дополнительно проводился поиск по цитированиям для выявления
важных исследований, которые могли быть пропущены. Списки литературы
включенных исследований были вручную проверены на наличие других
потенциально релевантных источников.

\subsection{Стратегия поиска}

\begin{table}[h!]
  \centering
  \begin{footnotesize}
  \caption{Извлекаемая информация}
  \label{tab:questions}
  \renewcommand{\arraystretch}{1.3}
  \begin{tabular}{|p{5cm}|p{9cm}|}
    \hline
    \multicolumn{2}{|c|}{\textbf{Сведения о публикации}} \\
    \hline
    Название исследования & - \\
    \hline
    Автор(ы) & Кто автор(ы) исследования/документа? \\
    \hline
    Год публикации & В каком году было опубликовано исследование? \\
    \hline
    Тип публикации & Является ли документ эмпирическим исследованием? \\
    \hline
    Страна происхождения & В какой стране было выполнено исследование? \\
    \hline

    \multicolumn{2}{|c|}{\textbf{Общие сведения}} \\
    \hline
    Цели/назначение & Каковы цели исследования/документа? \\
    \hline
    Дизайн исследования & Каков дизайн исследования/документа? \\
    \hline
    Контекст исследования & Какую реальную или воображаемую систему описывает исследование? \\
    \hline
    Популяция & Как описывается многоагентная система? \\
    \hline

    \multicolumn{2}{|c|}{\textbf{Содержательная часть}} \\
    \hline
    Методология & Какова предметная область? \\
    \cline{2-2}
     & Как моделируется предметная область? \\
    \cline{2-2}
     & Описание алгоритма \\
    \cline{2-2}
     & Является ли алгоритм специфичным для области? \\
    \hline
    Оценка результатов & Как определяется «успех»? \\
    \cline{2-2}
     & Как оценивается алгоритм? \\
    \cline{2-2}
     & Каковы преимущества алгоритма? \\
    \hline
    \end{tabular}
  \end{footnotesize}
\end{table}

Стратегия поиска была тщательно разработана с целью систематического выявления
научных публикаций, посвящённых вопросам кооперации в MAS.
Изначально был проведён разведочный поиск для выявления релевантной терминологии
и определения преобладающих исследовательских тем в данной области.
Этот этап имел решающее значение для точного выбора поисковых ключевых слов,
обеспечивая, чтобы последующие запросы были одновременно точными и исчерпывающими.
Для повышения эффективности поиска использовались продвинутые методы,
такие как булевы операторы и символы усечения.
Эти методы способствовали охвату широкого спектра лексических
вариаций и синонимичных терминов, тем самым расширяя рамки литературного обзора.
Использованные поисковые запросы:

\begin{itemize}
  \item \texttt{multi-agent systems AND conflict solving}
  \item \texttt{cooperated planning AND conflict solving}
  \item \texttt{multi-agent systems AND cooperated planning}
  \item \texttt{multi-agent systems AND (cooperated planning OR conflict solving)}
\end{itemize}

Итеративный характер поискового процесса позволял непрерывно уточнять стратегию,
адаптируясь к новой информации, выявленной в ходе обзора.
Помимо автоматизированных поисков в базах данных,
стратегия включала ручной анализ списков литературы в отобранных исследованиях.
Этот дополнительный подход сыграл ключевую роль в идентификации основополагающих работ
и дополнительных источников, которые могли быть упущены при исключительно электронном поиске.
Интеграция автоматических и ручных методов обеспечила всесторонний анализ
языковых нюансов, культурных особенностей и специфических вызовов в области MT.
В конечном итоге, данная комплексная стратегия поиска создала прочную основу для обзора,
обеспечив глубокое и многогранное понимание
различных методологий, направленных на обеспечение кооперации в MAS.
Подробное описание извлекаемой информации приведено в таблице~\ref{tab:questions}

\subsection{Выбор источников данных}

Заголовки и аннотации найденных статей просматривались одним рецензентом.
Если возникала неопределённость относительно включения статьи
на этапе первичного просмотра, она также включалась для полнотекстовой оценки.
Статьи, которые потенциально соответствовали
критериям включения/исключения (таблица~\ref{tab:criteria}),
переходили к этапу полнотекстового анализа.
Аналогично просмотру заголовков и аннотаций, статьи оценивались одним рецензентом.
Итоговые результаты поиска, включая количество просмотренных,
включённых и исключённых статей на каждом этапе,
были представлены в виде диаграммы потоков PRISMA (рис.~\ref{fig:PRISMA}),
в соответствии с рекомендациями PRISMA-ScR. Это визуально отображает
процесс отбора исследований и обеспечивает соответствие стандартам отчётности.

\subsection{Процесс извлечения данных}

Извлечение данных выполнялось одним членом исследовательской группы,
в соответствии с методологией JBI~\cite{afc61c6cf471416489e36a4bc382d3b9}.
Извлекаемые данные включали характеристики исследований и особенности
мультиагентных автономных систем, релевантные исследовательским вопросам\footnote{С промежуточными результатами можно ознакомится по ссылке
\url{https://gmatiukhin.site/articles/MAS-Conflict-Solving-Review/}}.

\subsection{Элементы данных}

\subsubsection{Синтез результатов}

\paragraph{Обобщение и суммирование результатов}

Первичный анализ включал как количественные, так и качественные методы.
Был проведен описательный числовой анализ исследований, охватывающий
количество работ, годы публикации, исследуемые популяции и ключевые методологии.
На основе исследовательских вопросов был проведен дедуктивный контент-анализ,
а также представлено нарративное описание табличных данных.

\paragraph{Представление результатов}

Результаты представлены в соответствии с рекомендациями PRISMA-ScR.
Диаграмма потока PRISMA (рис.~\ref{fig:PRISMA}) иллюстрирует процесс выбора исследований,
включая причины исключения на этапе полного текстового анализа.
Количественные результаты представлены в таблицах и сопровождаются
описательным анализом, согласующимся с исследовательскими вопросами.

\section{Результаты}

\subsection{Отбор источников данных}

В ходе поиска по базе ScienceDirect с использованием заданных поисковых запросов было выявлено 21770 исследований.
После удаления 4562 дубликатов (20,95\%) осталось 17203 уникальных записи.
Автоматизированные инструменты исключили 13371 исследование (82,37\%)
на основании их нерелевантности или недостаточной эмпирической/теоретической проработки вопросов,
связанных с мультиагентными системами, сотрудничеством и разрешением конфликтов.
Далее 3837 исследований (17,62\%) были отобраны для анализа на основе критериев включения/исключения.

На следующем этапе 76 публикаций (0,34\%) были отобраны для ручного анализа, из которых 75 были рассмотрены.
Из них 58 исследований (77,33\%) были исключены по следующим причинам:

\begin{itemize}
  \item 35 исследований (46,67\%) рассматривали исключительно теоретические основы;
  \item 12 исследований (16,00\%) не затрагивали ключевые аспекты сотрудничества в мультиагентных системах
    или их проектирования, включая моделирование домена или описание алгоритмов;
  \item 6 исследований (8,00\%) представляли собой систематические обзоры литературы без детализированных методологических аспектов;
  \item 5 исследований (6,67\%) были библиографическими обзорами и не содержали оригинальных эмпирических данных.
\end{itemize}

В результате в данный обзор было включено 17 исследований (0,07\% от общего числа записей).
Дополнительно с помощью поиска по цитированию были выявлены ещё 6 исследований, однако лишь 2 из них были включены в анализ.
В совокупности 19 исследований обеспечили релевантные эмпирические данные, теоретические основы,
а также представления о методах разрешения конфликтов и алгоритмах сотрудничества в мультиагентных системах,
включая терминологию, специфичную для данной области.

\begin{figure}
  \centering
  \includegraphics[width=0.8\linewidth]{PRISMA.png}
  \caption{PRISMA Flow Diagram}
  \label{fig:PRISMA}
\end{figure}

\subsubsection{Анализ временного распределения}

Анализ временной шкалы публикаций показывает стабильный, хотя и медленный,
темп появления новых статей в период с 1988 по 2001 год (рис.~\ref{fig:articles-by-year}).
Затем наблюдается десятилетний разрыв до 2011 года,
после которого количество публикаций удвоилось по сравнению с 1988-2001 годами.
Наибольшее число статей (три) было опубликовано в 2021 году
--- последнем году, включённым в данный обзор.
Этот тренд отражает как давнюю историю развития данной научной области,
так и неизменный интерес к ней на протяжении многих лет.

\begin{figure}
  \centering
  \includegraphics[width=0.8\linewidth]{articles_by_years.png}
  \caption{Статьи по годам}
  \label{fig:articles-by-year}
\end{figure}

\subsubsection{Географическое распределение публикаций}

С учётом совместных исследований лидером в данной области является США,
на долю которых приходится семь публикаций,
что составляет более трети всех рассмотренных работ (рис.~\ref{fig:countries-of-origin}).
На втором месте находится Испания с тремя статьями.
Китай, Австралия, Израиль и Чехия внесли по два исследования каждая.
В то же время Индия, Италия, Португалия, Словакия, Бельгия, Оман, Франция, Тунис, Турция и Нидерланды представлены одной публикацией каждая.

\begin{figure}
  \centering
  \includegraphics[width=0.8\linewidth]{countries_of_origin.png}
  \caption{Распределение по странам}
  \label{fig:countries-of-origin}
\end{figure}

\subsection{Определение текущих тенденций}

\begin{figure}
  \centering
  \includegraphics[width=0.8\linewidth]{co-occurence.png}
  \caption{Совместное использование ключевых слов}
  \label{fig:co-occurence}
\end{figure}

Анализируя данные, извлеченные из статей, можно выделить четыре основные области (рис.~\ref{fig:co-occurence}):

\paragraph{Математика}

Этот кластер (красный) акцентирует внимание на математическом моделировании, методах оптимизации и проектировании алгоритмов.
Существуют сильные связи между <<математикой>> и <<математическим анализом>>,
отражающие теоретические основы, а также между <<математикой>> и <<алгоритмом>>, что подчеркивает вычислительные стратегии.

\paragraph{Информатика}

Этот кластер (зеленый) сосредоточен на архитектуре систем, обработке данных и распределенных системах.
Значимые связи включают <<информатику>> с <<распределенными вычислениями>> и <<базой данных>>,
что иллюстрирует вычислительные методы для обработки данных в ресурсоемких задачах.

\paragraph{Искусственный интеллект}

Этот кластер (синий) охватывает интеллектуальные системы, принятие решений и оптимизацию процессов.
Сильные ассоциации наблюдаются между <<искусственным интеллектом>> и <<мультиагентными системами>>,
а также между <<ИИ>> и <<управлением процессами>>.

\paragraph{Инжиниринг и разработка систем}

Этот кластер (желтый) сосредоточен на внедрении и системной инженерии,
противопоставляя себя более теоретическим кластерам, описанным выше.
Существенные связи связывают <<системную инженерию>> с <<управлением задачами>> и <<математический анализ>> с <<управлением задачами>>,
подчеркивая интеграцию аналитических методов в проектирование систем.

Однако самые сильные связи --- между кластерами, особенно между математическими основами и вычислительными системами.
Связи, такие как <<математика>> и <<информатика>>, показывают, что математика является основой информатики,
в то время как <<алгоритм>> и <<искусственный интеллект>> подчеркивают интеграцию алгоритмических подходов в ИИ.
Междисциплинарные связи между <<распределёнными вычислениями>> и <<мультиагентными системами>> подчеркивают распределённые подходы в кооперативном планировании,
а связи между <<инженерией>> и <<искусственным интеллектом>> акцентируют внимание на том,
что, хотя рассматриваемые статьи в основном касаются теоретических приложений,
реальные потребности также находятся в центре внимания этих исследований.
Это не вызывает удивления,
поскольку оно отражает естественный порядок вещей в исследуемой области,
а именно — в области информатики.

\subsection{Синтез результатов}

\subsubsection{Моделирование проблемы}

\paragraph{Представление домена}

Алгоритмические подходы к разрешению конфликтов и кооперативному планированию в MAS
обычно основываются на определении структурированных доменов, в которых взаимодействуют агенты.
Эти домены моделируются с использованием формальных представлений,
таких как графы ограничений~\cite{SHARON201540,SEMIZ2021220},
иерархические сетевые задачи~\cite{FRANKOVIC20017} и архитектуры blackboard~\cite{DURFEE1988268}.
Подобные формализмы способствуют четкому определению целей,
ограничений и взаимозависимостей между агентами~\cite{GROSZ1996269}.
В качестве математических моделей часто используются задачи удовлетворения ограничений,
которые позволяют учитывать требования к принятию решений и накладываемые ограничения~\cite{KOMENDA201476}.
Благодаря этим формализмам агенты могут работать автономно, обеспечивая согласованность с общими целями.

\paragraph{Представление состояний и целей}

Подходы, основанные на моделировании состояний, позволяют агентам отслеживать их прогресс и динамически оценивать изменения в окружающей среде.
Проблемы, как правило, формулируются в виде переходов между состояниями,
где действия определяются для достижения заданных целей~\cite{CHOUHAN2015396}.
Агенты поддерживают локальные представления среды,
дополненные общими ограничениями и возможностями для согласования действий на глобальном уровне~\cite{ROSENSCHEIN1988187}.
Цели могут определяться явно как конечные состояния или неявно через функции полезности,
оценивающие эффективность выполнения задач~\cite{MA2021103823}.
Кроме того, модели часто включают вероятностные методы рассуждения для учета неопределенности,
что повышает устойчивость и надежность принятия решений~\cite{WU2011487}.

\paragraph{Временное и динамическое моделирование}

С учетом динамической природы мультиагентных систем многие подходы интегрируют временные модели
для управления изменяющейся средой~\cite{MA2021103823,LU2014215}.
Для структурирования последовательности задач и координации выполнения
в изменяющихся условиях используются рабочие процессы и событийно-ориентированные фреймворки.
Временные ограничения задают границы временных окон выполнения задач,
обеспечивая синхронизацию между агентами.
Динамические обновления посредством инкрементальных механизмов планирования позволяют адаптироваться в реальном времени,
как это наблюдается в методах с итеративным уточнением и поэтапной корректировкой планов~\cite{SEMIZ2021220}.
Эти методологии позволяют системам реагировать на непредвиденные изменения без необходимости полного перепланирования.

\subsubsection{Стратегии разрешения конфликтов}

\paragraph{Удовлетворение и расслабление ограничений}

Алгоритмы, такие как Conflict-Based Search,
решают конфликты путем итеративного уточнения решений
за счет добавления или ослабления ограничений~\cite{SHARON201540}.
Эти стратегии опираются на пошаговые оценки для эффективного разрешения конфликтов при сохранении оптимальности.
В более динамичных условиях инкрементальные корректировки позволяют системам
устранять неожиданные конфликты в реальном времени без необходимости полного пересмотра планов~\cite{KOMENDA201476}.

\paragraph{Переговоры и торги}

Подходы, основанные на переговорах, используют механизмы распределения задач,
включая модели торгов и аукционов,
для распределения ресурсов и задач между агентами~\cite{GHARRAD2021108282,FRANKOVIC20017,RABELO1994303}.
Эти методы акцентируют кооперативные взаимодействия,
в которых агенты итеративно уточняют соглашения на основе приоритетов и ограничений.
Гибкость в перераспределении задач способствует масштабируемости и способности адаптироваться к изменяющимся условиям.
Динамические корректировки и циклы пересмотра договоренностей обеспечивают устранение возникающих конфликтов в ходе выполнения задач.

\paragraph{Аргументация и мета-рассуждение}

Некоторые фреймворки включают методы аргументации,
позволяя агентам участвовать в структурированных диалогах для разрешения конфликтов~\cite{PAJARESFERRANDO201322,FERRANDO20171}.
Системы аргументации оценивают конкурирующие утверждения,
выявляют несоответствия и приходят к соглашениям через процессы мета-рассуждения.
Этот подход особенно эффективен в сценариях,
требующих объяснительного обоснования решений или согласованных действий в рамках коллективного контекста.
Моделирование обновления убеждений и корректировки предпочтений
в аргументационных механизмах способствует прозрачному принятию решений и адаптивности.

\subsubsection{Механизмы кооперации}

\paragraph{Распределенный и децентрализованный контроль}

Методы распределенного планирования составляют основу кооперативных мультиагентных систем.
Эти подходы избегают централизованного управления~\cite{DURFEE1988268,GRASTIEN2020103271,MA2021103823}
в пользу децентрализованных структур, где агенты действуют независимо,
координируясь через общие коммуникационные протоколы~\cite{ROSENSCHEIN1988187}.
Такая структура повышает масштабируемость, отказоустойчивость и надежность системы,
снижая зависимость от централизованных координаторов.
Протоколы обеспечивают обмен ограничениями и обновлениями, позволяя агентам синхронизировать действия и эффективно использовать ресурсы.

\paragraph{Декомпозиция задач и иерархии}

Методы иерархической декомпозиции задач позволяют разделять сложные проблемы на более мелкие подзадачи,
обеспечивая их параллельное выполнение и улучшенный контроль~\cite{JUNG1999149}.
Эти подходы используют многоуровневые структуры планирования,
где высокоуровневые цели определяют низкоуровневые действия.
Иерархии способствуют разрешению конфликтов за счет изоляции зависимостей внутри групп задач,
что упрощает управление взаимодействиями агентов.
Такая структура также поддерживает масштабируемость и модульное планирование~\cite{RODRIGUEZ201113005},
делая ее подходящей для крупных систем.

\paragraph{Динамические обновления и адаптация}

Механизмы динамической адаптации обеспечивают способность мультиагентных систем реагировать на изменения в окружающей среде.
Циклы обратной связи и методы итеративного обучения позволяют агентам непрерывно оценивать производительность
и корректировать планы на основе наблюдаемых результатов~\cite{LU2014215}.
Методы восстановления планов и инкрементальной корректировки повышают устойчивость системы,
сводя к минимуму сбои в ходе выполнения задач.
Эти механизмы особенно важны в динамичных доменах, где условия изменяются непредсказуемо.

\subsubsection{Оценка эффективности}

Оценка эффективности представленных методологий выявляет их сильные и слабые стороны.
Методы удовлетворения ограничений отличаются математической строгостью и масштабируемостью,
что делает их эффективными в структурированных доменах,
где ограничения могут быть четко заданы~\cite{STOLBA2017175}.
Однако в условиях высокой динамичности или неопределенности они могут требовать дополнительных адаптивных слоев~\cite{SHARON201540}.
Переговорные подходы демонстрируют гибкость и адаптивность за счет динамического распределения задач.
Итеративная корректировка и пересмотр соглашений делают их пригодными для открытых и изменяющихся сред.
Тем не менее, их зависимость от коммуникационных протоколов может вызывать накладные расходы в условиях ограниченной пропускной способности или задержек~\cite{PAJARESFERRANDO201322}.
Аргументационные фреймворки эффективны в областях, требующих обоснования решений,
обеспечивая прозрачность и доверие в коллективном принятии решений.
Однако их вычислительная сложность может быть ограничивающим фактором в масштабных системах с многочисленными агентами~\cite{FERRANDO20171}.
Кооперативные механизмы, основанные на распределенном управлении и иерархической декомпозиции,
демонстрируют высокую масштабируемость и отказоустойчивость,
что делает их эффективными в больших децентрализованных системах~\cite{MA2021103823}.
Однако они могут требовать сложных протоколов координации для обеспечения согласованности, особенно в системах с разнородными агентами~\cite{JUNG1999149}.
Методы динамической адаптации повышают устойчивость и оперативность,
делая их незаменимыми в условиях неопределенности и изменений~\cite{LU2014215}.
Однако их эффективность часто зависит от качества механизмов обратной связи
и способности балансировать исследование новых решений и использование имеющегося опыта.

\section{Обсуждение}

\subsection{Обобщение данных}

Этот обзор рассматривает стратегии разрешения конфликтов и механизмы кооперации в мультиагентных системах,
фокусируясь на моделировании, методах решения и взаимодействии.
Формальные представления, такие как задачи удовлетворения ограничений,
иерархические сети задач и временные модели, служат структурной основой~\cite{GROSZ1996269,KOMENDA201476}.
Эти методы учитывают ограничения и неопределенности, поддерживая координацию через основанное на состояниях рассуждение и обновления~\cite{STOLBA2017175,LU2014215}.

Разрешение конфликтов опирается на удовлетворение ограничений, переговоры и аргументационные структуры.
Конфликтно-ориентированный поиск уточняет решения для динамических сред~\cite{SHARON201540}.
Переговоры и торговля улучшают распределение ресурсов,
но увеличивают коммуникационные издержки~\cite{PAJARESFERRANDO201322,FRANKOVIC20017}.
Аргументационные методы повышают прозрачность, но страдают от проблем масштабируемости~\cite{FERRANDO20171}.

Кооперация достигается через распределенное управление и иерархии задач,
обеспечивая масштабируемость и отказоустойчивость~\cite{MA2021103823,JUNG1999149}.
Динамическая адаптация повышает устойчивость с помощью итеративного обучения и обратной связи~\cite{SEMIZ2021220}.

Эффективность методов зависит от условий: модели ограничений успешны в структурированных средах,
переговоры гибки в динамических сценариях, аргументация повышает объяснимость решений.
Однако остаются вызовы масштабируемости, адаптивности и вычислительных затрат,
требующие усовершенствованных критериев оценки и универсальных методологических основ.
Дальнейшие исследования должны устранить эти ограничения для создания надежных и масштабируемых решений.

\subsection{Ограничения}

Можно выделить несколько потенциальных ограничений.  

\subsubsection{Отсутствие унифицированных стандартов оценки}

Отсутствие единых кросс-доменных критериев оценки в различных исследованиях
создает проблему для проведения содержательных сравнений.
Например, одни работы делают акцент на вычислительной эффективности,
тогда как другие отдают приоритет адаптивности или стабильности.

\subsubsection{Зависимость от данных, полученных на основе симуляций}

Значительная часть представленных в рассмотренных исследованиях данных получена из симуляций,
а не из реальных внедрений. Хотя симуляции обеспечивают контролируемые условия тестирования,
они могут не полностью отражать сложности развертывания мультиагентных систем в динамических реальных средах.  

Устранение этих ограничений в будущих исследованиях за счет расширения критериев включения,
стандартизации метрик и интеграции реальных кейс-стадий позволит повысить полноту и актуальность обзора.
