\chapter*{Введение}
\addcontentsline{toc}{chapter}{Введение}

\section*{Актуальность работы}

С ростом сложности распределённых систем и увеличением числа взаимодействующих
интеллектуальных компонентов возникает потребность
в эффективных методах планирования и координации действий.
Мультиагентные системы находят широкое применение в таких сферах,
как логистика, управление беспилотными устройствами,
моделирование и робототехника.
Однако одной из ключевых проблем остаётся разрешение противоречий между целями агентов,
действующих в общем пространстве. 
Существующие решения либо ориентированы на строго кооперативные сценарии,
либо не масштабируются при увеличении числа агентов.
Это обуславливает необходимость разработки архитектур и алгоритмов,
способных учитывать несовпадающие цели,
обеспечивать частичную автономию агентов и устойчивость системы
при частичной недостижимости целей.
Настоящая работа направлена на создание планировщика, способного решать подобные задачи.

\section*{Цели работы}

Целью работы является разработка архитектуры
и программной реализации планировщика,
обеспечивающего согласование действий агентов
в мультиагентной системе при наличии потенциально конфликтующих целей.

\section*{Задачи работы}

Для достижения поставленной цели необходимо решить следующие задачи:
\begin{itemize}
  \item Провести обзор современных методов планирования и кооперации в мультиагентных системах;
  \item Проанализировать подходы к разрешению конфликтов между агентами;
  \item Разработать и реализовать одноагентный планировщик как основу для мультиагентного решения;
  \item Спроектировать мультиагентную архитектуру,
    обеспечивающую взаимодействие агентов, оценку достижимости целей и обмен планами;
  \item Реализовать методы устойчивого планирования с учётом частичной недостижимости целей;
  \item Оценить корректность и применимость разработанного подхода на демонстрационных сценариях.
\end{itemize}

\section*{Апробация работы}

В ходе выполнения работы были получены результаты, представленные на
Всероссийской конференции с международным участием
<<Информационно"=телекоммуникационные технологии
и математическое моделирование высокотехнологичных систем>> (Москва, РУДН, 2025 г.).

\section*{Публикации}

По теме выпускной квалификационной работы бакалавра была опубликована работа~\cite{ittmm}.

\section*{Структура работы}

Работа состоит из введения, трех глав, заключения и списка литературы.
Во введении отражена проблематика выбранной
темы, актуальность этой темы в современном мире,
поставлены цели и задачи данной работы.
Первая глава посвящена систематическому обзору существующих методов планирования, кооперации и разрешения конфликтов в мультиагентных системах. Третья глава описывает реализацию одноагентного планировщика: архитектуру, внутренние структуры и принципы работы. В четвёртой главе рассматривается переход к мультиагентной системе: уточнение целей, межагентная коммуникация и устойчивость. Заключение содержит основные выводы и направления дальнейших исследований.

\section*{Справочная информация}

\subsection*{Мультиагентные системы}

Мультиагентные системы (MAS, \textit{Multi-Agent System}) играют центральную роль в искусственном интеллекте,
обеспечивая децентрализованное принятие решений в таких областях, как автономные
транспортные средства, интеллектуальные энергосистемы и робототехника~\cite{Torre_o_2017}.
Их масштабируемость, адаптивность и отказоустойчивость делают их незаменимыми
для сложных, динамических сред, требующих кооперации и разрешения конфликтов.

Несмотря на достижения, остаются проблемы, связанные с эффективной координацией
и разрешением конфликтов~\cite{galesloot2024factoredonlineplanningmanyagent}
\cite{zhang2014formalanalysisrequiredcooperation}. Такие методы, как задачи удовлетворения
ограничений (CSP, \textit{Constraint Satisfaction Problem})~\cite{KOMENDA201476} и модели, основанные на переговорах
\cite{GROSZ1996269,RABELO1994303}, демонстрируют потенциал, но часто сталкиваются
с проблемами масштабируемости в реальных условиях. Кроме того, отсутствие единых
оценочных метрик затрудняет сравнительный анализ.


\subsection*{Планирование}

Планирование в области искусственного интеллекта представляет собой процесс автоматического построения последовательности действий, необходимых для достижения цели. Классическое планирование предполагает полное знание среды, детерминированность действий и статичность мира. Однако в реальных приложениях — особенно в мультиагентных системах — нередко возникают ситуации с неполной информацией, конкурирующими интересами и необходимостью адаптации плана в процессе выполнения. Это требует разработки более гибких моделей, способных учитывать множественность целей и агентов, действующих независимо.

\subsection*{Планировщики}

Планировщики~\cite{ghallab2004automated} --- это системы, используемые для автоматического создания последовательности действий, необходимых для достижения заданной цели в определенной среде. Они широко применяются в различных областях, таких как робототехника, логистика, искусственный интеллект и управление задачами. Основная задача планировщика заключается в нахождении оптимального или приемлемого плана, который переводит систему из начального состояния в целевое состояние, удовлетворяя при этом заданным ограничениям и условиям.

\subsection*{Planning Domain Definition Language}

PDDL (Planning Domain Definition Language)~\cite{mcdermott1998pddl}~\cite{gerevini2006pddl3}
--- это язык описания задач планирования, используемый в области искусственного интеллекта.
Он позволяет формализовать модель мира и определить цели, которых необходимо достичь.
Обычно описание задачи в PDDL состоит из двух основных файлов:
\textit{файла домена} и \textit{файла задачи}.
Это деление позволяет отделить общую модель мира (правила, действия, объекты и свойства)
от конкретной задачи, которая решается в рамках этого мира.

\begin{enumerate}
\item \textbf{Файл домена} описывает, какие действия возможны в рассматриваемом мире,
  какие предикаты используются для описания состояния,
    и как действия изменяют это состояние.
    Ниже приведён пример PDDL-домена для маленького склада:

  \begin{minted}{lisp}
(define (domain mini-warehouse)
  (:requirements :strips :typing :quantified-preconditions)
  (:types robot box location)
  (:predicates
    (at ?r - robot ?l - location)
    (box-at ?b - box ?l - location)
    (carrying ?r - robot ?b - box)
    (clean ?l - location)
    (adjacent ?l1 - location ?l2 - location) )
  (:action move
    :parameters (?r - robot ?from - location ?to - location)
    :precondition (and (at ?r ?from) (adjacent ?from ?to))
    :effect (and (not (at ?r ?from)) (at ?r ?to)) )
  (:action pick-up
    :parameters (?r - robot ?b - box ?l - location)
    :precondition (and (at ?r ?l) (box-at ?b ?l))
    :effect (and (carrying ?r ?b) (not (box-at ?b ?l))) )
  (:action drop
    :parameters (?r - robot ?b - box ?l - location)
    :precondition (and (at ?r ?l) (carrying ?r ?b))
    :effect (and (box-at ?b ?l) (not (carrying ?r ?b))) )
  (:action clean-nearby
    :parameters (?r - robot ?l - location)
    :precondition (at ?r ?l)
    :effect (forall (?x - location)
      (when (adjacent ?l ?x) (clean ?x))))
)
  \end{minted}

    В этом примере моделируется поведение роботов на складе:
    перемещение между локациями, управление коробками и очистка территорий.
    Используются возможности языка PDDL:
    \begin{itemize}
      \item \texttt{:strips},
      \item \texttt{:typing},
      \item \texttt{:quantified-preconditions}.
    \end{itemize}
    Определены три типа объектов: роботы, коробки, локации.
    Предикаты задают положение роботов и коробок, факт переноски, чистоту и смежность локаций.
    Действия позволяют роботу: перемещается между соседними локациями,
    поднимать коробку из текущей локации, опускать коробку в текущей локации, очищать все смежные с текущей локации.

\item \textbf{Файл задачи} указывает,
  какие объекты существуют в данной конкретной задаче,
    каково начальное состояние и какое целевое состояние требуется достичь.
    Вот пример соответствующего файла задачи:

  \begin{minted}{lisp}
(define (problem mini-a)
  (:domain mini-warehouse)
  (:objects
    r1 - robot
    b1 b2 - box
    l1 l2 l3 - location)
  (:init
    (at r1 l1)
    (box-at b1 l2)
    (box-at b2 l3)
    (adjacent l1 l2)
    (adjacent l2 l3)
    (adjacent l1 l3))
  (:goal (and
    (box-at b1 l3)
    (clean l2)))
)
  \end{minted}

  Здесь задача использует домен \texttt{mini-warehouse}. 
  В задаче заданы: один робот, две коробки, три локации.
  Робот находится в \texttt{l1}, коробки \texttt{b1} и \texttt{b2} -- в \texttt{l2} и \texttt{l3} соответственно.
  Все локации попарно смежны.
  Целью является: коробка \texttt{b1} находится в локации \texttt{l3} и локация \texttt{l2} должна быть очищена.
\end{enumerate}

Такое разделение обеспечивает переиспользуемость:
один и тот же домен можно использовать для множества различных задач.
Кроме того, это упрощает тестирование и отладку планирующих систем.
Благодаря своей формальной структуре PDDL поддерживает автоматическую проверку корректности задач
и их совместимости с определением домена,
что делает его мощным инструментом для разработки и исследования алгоритмов планирования.

Решением для такой этой конкретной проблемы будет следующая последовательность действий:
\begin{minted}{lisp}
(
  (move r1 l1 l2)
  (pick-up r1 b1 l2)
  (move r1 l2 l3)
  (drop r1 b1 l3)
  (clean-nearby r1 l3)
)
\end{minted}

Задача планировщика --- принять на вход описание домена и задачи, и вывести последовательность действий для ее решения.
